
\documentclass[handout]{beamer}
%\documentclass{beamer}

\mode<presentation>
{
  \usetheme{Warsaw}
  \setbeamercovered{transparent}
}

\usepackage[english]{babel}
\usepackage[latin1]{inputenc}
\usepackage{times}
\usepackage[T1]{fontenc}

\title[MetagenomeDB Primer]{MetagenomeDB Primer}

\date[November 5, 2010]

\subject{Talks}

\AtBeginSection[]
{
	\begin{frame}<beamer>{Outline}
		\tableofcontents[currentsection, currentsubsection]
	\end{frame}
}

\AtBeginSubsection[]
{
	\begin{frame}<beamer>{Outline}
		\tableofcontents[currentsection, currentsubsection]
	\end{frame}
}

%\beamerdefaultoverlayspecification{<+->}
\setbeamertemplate{navigation symbols}{}

\def\prompt{>\hspace{-1pt}>\hspace{-1pt}>\/}
\newcommand{\comment}[1]{{\raggedright $\rightarrow$~\textit{\textsf{#1}}}}

\begin{document}

\begin{frame}
	\titlepage
\end{frame}

\begin{frame}{Outline}
	\tableofcontents
\end{frame}

%%%%%%%%%%%%%%%%%%%%%%%%%%%%%%%%%%%%%%%%%%%%%%%%%%%%%%%%%%%%%%%%%%%%%%%%%%%%%%%%

\section{Introduction}

\begin{frame}{What is MetagenomeDB?}
	\textsc{MetagenomeDB} is a Python\footnote{\url{http://www.python.org/}}-based library designed to easily store, retrieve and annotate metagenomic sequences. It provide an API to create and modify two types of objects, namely sequences and collections. Behind the scene, all data are handled to a MongoDB\footnote{\url{http://www.mongodb.org/}} database to ensure reliability and speed.
\end{frame}

\begin{frame}{Sequences and Collections}
	`Sequences' are any sequence a metagenomic project generate; mainly, reads and contigs. Sequences can be annotated and related to other sequences (to represent similarities, or when a sequence is part of another sequence; e.g., reads that are part of a contig).

	\bigskip
	`Collections' are sets of sequences. Collections can also contains sub-collections to represent hierarchies (e.g., replicate sets of reads, or alternative assemblies). Collections can be annotated and related to other sequences and collections.
\end{frame}

\section{Using MetagenomeDB}

\begin{frame}{Content of the MetagenomeDB toolkit}
	\textsc{MetagenomeDB} has two components:
	\begin{itemize}
		\item a library, which can be used in your own Python scripts to access the database, and
		\item a set of command-line tools, which automatize the addition, annotation and deletion of sequences and collections in the database.
	\end{itemize}

	\medskip
	Among the command-line tools you will find utilities to import BLAST and FASTA alignment results, FASTA-formatted sequences, and ACE reads-to-contigs mappings.
\end{frame}

\begin{frame}[fragile]
	\frametitle{Loading the MetagenomeDB library}
	Once installed, \textsc{MetagenomeDB} can be used in any Python program as a module using the \texttt{import} statement:

	\begin{verbatim}
		import MetagenomeDB
	\end{verbatim}

	Alternatively, you can provide an alias to shorten the name of the library:

	\begin{verbatim}
		import MetagenomeDB as mdb
	\end{verbatim}
\end{frame}

%\section{Test cases}

\end{document}
